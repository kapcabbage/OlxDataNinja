\documentclass[a4paper,11pt]{article}
\usepackage{latexsym}
\usepackage[polish]{babel}
\usepackage[utf8]{inputenc} 
\usepackage[MeX]{polski}
\usepackage{nicefrac}
\usepackage{graphicx}
\usepackage{array}
\usepackage{amsmath}
\usepackage{graphicx}
\usepackage[colorinlistoftodos]{todonotes}
\usepackage{fancyhdr}
\usepackage{listings}
\usepackage[T1]{fontenc}
\usepackage{inconsolata}
\usepackage{color}
\usepackage{hyperref}
\usepackage{float}

\author{Phi Long Nguyen 116834 \\ Kajetan Gulbierz 134494}

\title{MMDS Challenge\\
\large{{\bf Raport I}  }} 

\date{24 listopada  2017}

\begin{document}

\maketitle 

\section{{Analiza danych ogłoszeń}}
\paragraph{Ogłoszenia} dostarczone są w formie plików, gdzie każdy z nich reprezentuje jeden miesiąc. W pliku zawierają się ogłoszenia z tego miesiąca. Każde ogłoszenie ma ściśle określoną strukturę, dzięki czemu jesteśmy w stanie je analizować. Celem zadania jest zrozumienie zjawiska jakiego dotyczą dane. 

\subsection{Histogramy}
\paragraph{Wykresy} mają za zadanie pokazać trendy, jakie się utrzymują na przestrzeli roku (histogram sprzedaży). Oprócz tego pozwalają zobaczyć ile ogłoszeń miało duża liczbę wyświetleń oraz dużą liczbę odpowiedzi.\


\subsubsection{Histogram dla sprzedaży} 
Rysunek \ref{fig:automat} pokazuje histogram liczby sprzedaży ogółem dla każdego z miesięcy. Można zauważyć, że ogół sprzedaży jest największy dla pierwszego miesiąca w roku i stopniowo spada. Taki stan rzeczy utrzymuje się mniej więcej do sierpnia, gdzie liczba gwałtownie się podnosi. Na takie wyniki może wpływać fakt, że wrzesień jest miesiącem rozpoczynającym rok szkolny. Najmniejsze zaś wyniki można odnotować w okresie wakacyjnym.  

\begin{figure}[H]
	\centering
	\includegraphics[scale=0.25]{./sold.png}
	\caption{\label{fig:automat}Histogram sprzedaży}
\end{figure}

\subsubsection{Histogram liczby wyświetleń} 
Rysunek \ref{fig:wyswietlenia} pokazuje wykres liczby ogłoszeń mających mniej niż X wyświetleń. Przedziały X są oznaczone na osi-x.

\begin{figure}[H]
	\centering
	\includegraphics[scale=0.25]{./sold.png}
	\caption{\label{fig:wyswietlenia}Wykres wyświetleń}
\end{figure}

\subsubsection{Histogram liczby odpowiedzi} 
Rysunek \ref{fig:odpowiedzi} pokazuje wykres liczby ogłoszeń mających mniej niż X odpowiedzi. Przedziały X są oznaczone na osi-x.

\begin{figure}[H]
	\centering
	\includegraphics[scale=0.25]{./sold.png}
	\caption{\label{fig:odpowiedzi}Wykres wyświetleń}
\end{figure}

\subsubsection{Wykres korelacji} 
Rysunek \ref{fig:korelacja} pokazuje zależność między wyświetleniami, odpowiedzią a sprzedażą w danym miesiącu. Można zauważyć o wiele większą rozpiętość zarówno wyświetleń jak i odpowiedzi dla ogłoszeń, które nie zostały zakończone sprzedażą w stosunku do tych, którym sprzedaż się powiodła. 
 
\begin{figure}[H]
	\centering
	\includegraphics[scale=0.35
	]{./correlation.png}
	\caption{\label{fig:korelacja}Wykres korelacji}
\end{figure}

\section{Analiza danych zapytań} 
\paragraph{Pliki zapytań} reprezentują liczbę sesji dla danych kategorii. W ramach sesji tworzone są zapytania. Liczba sesji \textit{de facto} oznacza ile aktywnych użytkowników w danym momencie wystosowało dane zapytanie.


\subsection{Cykliczność tygodniowa}
Badanie cykliczności liczby zapytań ma za zadanie pomóc dostrzec wzorzec jaki spełniają dane w tygodniu. Na rysunku \ref{fig:day} można zauważyć wzrost liczby zapytań w ostatnie dni tygodnia. Najmniejszy zaś wynik przypada w piątek, co może oznaczać, że początek weekendu użytkownicy spędzają inaczej, niż na przeglądaniu ofert;
   

\begin{figure}[H]
	\centering
	\includegraphics[scale=0.45]{./day.png}
	\caption{\label{fig:day}Wykres liczby zapytań w danym dniu tygodnia}
\end{figure}

\subsection{Wykres popularności}
Rysunek \ref{fig:iphoine} pokazuje jak zmieniała się popularność - w tym wypadku dla kategorii \textit{iPhone}. Można zauważyć, znaczny wzrost liczby wyszukań w okresie wakacyjnym. Sporą liczbę zapytań można również zauważyć w grudniu - można przypuszczać, że ma to związek z okresem świątecznym.
   

\begin{figure}[H]
	\centering
	\includegraphics[scale=0.8]{./iphone.png}
	\caption{\label{fig:iphoine}Wykres popularności zapytania dla iPhone}
\end{figure}

\end{document}  