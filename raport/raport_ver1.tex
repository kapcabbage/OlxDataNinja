\documentclass[a4paper,11pt]{article}
\usepackage{latexsym}
\usepackage[polish]{babel}
\usepackage[utf8]{inputenc} 
\usepackage[MeX]{polski}
\usepackage{nicefrac}
\usepackage{graphicx}
\usepackage{array}
\usepackage{amsmath}
\usepackage{graphicx}
\usepackage[colorinlistoftodos]{todonotes}
\usepackage{fancyhdr}
\usepackage{listings}
\usepackage[T1]{fontenc}
\usepackage{inconsolata}
\usepackage{color}
\usepackage{hyperref}
\usepackage{float}

\author{Phi Long Nguyen 116834 \\ Kajetan Gulbierz}

\title{MMDS Challenge\\
\large{{\bf Raport I}  }} 

\date{24 listopada  2017}

\begin{document}

\maketitle 

\section{Histogramy}
\paragraph{Wykresy} tyczą się plików \textit{ads}. Mają one za zadanie pokazać trendy, jakie się utrzymują na przestrzeli roku (histogram sprzedaży). Oprócz tego pozwalają zobaczyć ile ogłoszeń miało duża liczbę wyświetleń oraz dużą liczbę odpowiedzi.\


\subsection{Histogram dla sprzedaży} 
Rysunek \ref{fig:automat} pokazuje histogram liczby sprzedaży ogółem dla każdego z miesięcy. Można załuważyć, że ogół sprzedaży jest największy dla pierwszego miesiąca w roku i stopniowo spada. Taki stan rzeczy utrzymuje się mniej więcej do sierpnia, gdzie liczba gwałtownie się podnosi. Na takie wyniki może wpływać fakt, że wrzesień jest miesiącem rozpoczynającym rok szkolny. Najmiejsze zaś wyniki można odnotować w okresie wakacyjnym.  

\begin{figure}[H]
	\centering
	\includegraphics[scale=0.25]{./sold.png}
	\caption{\label{fig:automat}Histogram sprzedaży}
\end{figure}

\subsection{Histogram liczby wyświetleń} 
Rysunek \ref{fig:wyswietlenia} pokazuje wykres liczby ogłoszeń mających mniej niż X wyświetleń. Przedziały X są oznaczone na osi-x.

\begin{figure}[H]
	\centering
	\includegraphics[scale=0.25]{./sold.png}
	\caption{\label{fig:wyswietlenia}Wykres wyświetleń}
\end{figure}

\subsection{Histogram liczby odpowiedzi} 
Rysunek \ref{fig:odpowiedzi} pokazuje wykres liczby ogłoszeń mających mniej niż X odpowiedzi. Przedziały X są oznaczone na osi-x.

\begin{figure}[H]
	\centering
	\includegraphics[scale=0.25]{./sold.png}
	\caption{\label{fig:odpowiedzi}Wykres wyświetleń}
\end{figure}

\end{document}  